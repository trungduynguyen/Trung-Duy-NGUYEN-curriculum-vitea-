%% start of file `template.tex'.
%% Copyright 2006-2013 Xavier Danaux (xdanaux@gmail.com).
%
% This work may be distributed and/or modified under the
% conditions of the LaTeX Project Public License version 1.3c,
% available at http://www.latex-project.org/lppl/.
%Version for spanish users, by dgarhdez

\documentclass[11pt,a4paper,roman]{moderncv}        % possible options include font size ('10pt', '11pt' and '12pt'), paper size ('a4paper', 'letterpaper', 'a5paper', 'legalpaper', 'executivepaper' and 'landscape') and font family ('sans' and 'roman')
\usepackage[spanish,es-lcroman]{babel}

%\usepackage{moderncv}
% moderncv themes
%\moderncvstyle{classic}                            % style options are 'casual' (default), 'classic', 'oldstyle' and 'banking'
%\moderncvcolor{green}                              % color options 'blue' (default), 'orange', 'green', 'red', 'purple', 'grey' and 'black'
%\renewcommand{\familydefault}{\sfdefault}         % to set the default font; use '\sfdefault' for the default sans serif font, '\rmdefault' for the default roman one, or any tex font name
%\nopagenumbers{}                                  % uncomment to suppress automatic page numbering for CVs longer than one page
% character encoding
%\usepackage[utf8]{inputenc}                       % if you are not using xelatex ou lualatex, replace by the encoding you are using
%\usepackage{CJKutf8}                              % if you need to use CJK to typeset your resume in Chinese, Japanese or Korean

% adjust the page margins
\usepackage[scale=0.75]{geometry}
%\setlength{\hintscolumnwidth}{3cm}                % if you want to change the width of the column with the dates
%\setlength{\makecvtitlenamewidth}{10cm}           % for the 'classic' style, if you want to force the width allocated to your name and avoid line breaks. be careful though, the length is normally calculated to avoid any overlap with your personal info; use this at your own typographical risks...

% personal data
\name{Trung-Duy}{NGUYEN}
\title{Letter of Motivation}                               % optional, remove / comment the line if not wanted
\address{Data Scientist at KPMG Deutschland}%{Frankfurt am Main, Germany}{}% optional, remove / comment the line if not wanted; the "postcode city" and and "country" arguments can be omitted or provided empty
%\phone[mobile]{000-000-000-000}                   % optional, remove / comment the line if not wanted
%\phone[fixed]{+2~(345)~678~901}                    % optional, remove / comment the line if not wanted
%\phone[fax]{+3~(456)~789~012}                      % optional, remove / comment the line if not wanted
\email{nguyentd@eurecom.fr}                               % optional, remove / comment the line if not wanted
%\homepage{www.johndoe.com}                         % optional, remove / comment the line if not wanted
%\extrainfo{additional information}                 % optional, remove / comment the line if not wanted
%\photo[64pt][0.4pt]{picture}                       % optional, remove / comment the line if not wanted; '64pt' is the height the picture must be resized to, 0.4pt is the thickness of the frame around it (put it to 0pt for no frame) and 'picture' is the name of the picture file
%\quote{Some quote}                                 % optional, remove / comment the line if not wanted

% to show numerical labels in the bibliography (default is to show no labels); only useful if you make citations in your resume
%\makeatletter
%\renewcommand*{\bibliographyitemlabel}{\@biblabel{\arabic{enumiv}}}
%\makeatother
%\renewcommand*{\bibliographyitemlabel}{[\arabic{enumiv}]}% CONSIDER REPLACING THE ABOVE BY THIS

% bibliography with mutiple entries
%\usepackage{multibib}
%\newcites{book,misc}{{Books},{Others}}
%----------------------------------------------------------------------------------
%            content
%----------------------------------------------------------------------------------
\begin{document}
%-----       letter       ---------------------------------------------------------
% recipient data
\recipient{Destinatario}{Departamento, Empresa}
\date{\today}
\opening{Estimado Destinatario,}
\closing{Muchas gracias por su tiempo e interés y reciba un cordial saludo.}
\enclosure[Adjunto]{CV}          % use an optional argument to use a string other than "Enclosure", or redefine \enclname
\makelettertitle

I am writing to apply for the tenure-track position in mathematics as advertised on the Employment Information in the Mathematical Sciences List. I am a graduate student at Harvard University working in Algebraic Combinatorics under the direction of Professor Stanton Lochs. I expect to complete my PhD by May 2017.

As my CV illustrates, I have a broad range of working experience in the industry; fortunately, most of the projects, which are in the past and on going, are immensely focused on researching and engaging state of the art machine learning algorithms into practice. I participated in the sentiment analysis project on Twitter and Starbucks dataset with Tenpoint7 which is the fast growing data consulting business based in Seattle, as a role of data analyst. In this project, I performed numerous natural language processing (NLP) techniques and machine learning algorithms to classify  users comments and reviews on giant social networks; at the end, the final analysis, which essentially influences business decisions and strategies, not only derived the adequate insight of customer behaviours, but also significantly increased up to twenty-seven percent of the revenue confirmed by Tenpoint7 partners. Recently, I had an opportunity to coporate in an academic project, which the aim is constructing a formal representation of cuisine recipes for machine learning models, with Professor Maurizio Fillipone at EURECOM, who is a famous and comprehensive researcher in Bayesian optimization and Gaussian process. In spite of researching alone, it turns out a valuable experience, which reinforces my capability of carrying out independently research and lateral thinking. I sucessfully designed an unique ingredients parser based semantics for recipes which particularly outperforms similar parsers on the field, which is one of the remarkable project outcomes.


\vspace{0.5cm}


\makeletterclosing

\end{document}


%% end of file `template.tex'.
